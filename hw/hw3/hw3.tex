\documentclass[12pt, notitlepage, final]{article} 

\newcommand{\name}{Vince Coghlan}

\usepackage{amsfonts}
\usepackage{amssymb}
\usepackage{amsmath}
\usepackage{latexsym}
\usepackage{enumerate}
\usepackage{amsthm}
\usepackage{nccmath}
\usepackage{setspace}
\usepackage[pdftex]{graphicx}
\usepackage{epstopdf}
\usepackage[siunitx]{circuitikz}
\usepackage{tikz}
\usepackage{float}
\usepackage{cancel}
\usepackage{pgfplots}
\usepackage{setspace}
\usepackage{overpic}
\usepackage{mathtools}
\usepackage{listings}
\usepackage{hyperref}
\usepackage{color}

\numberwithin{equation}{section}
\DeclareRobustCommand{\beginProtected}[1]{\begin{#1}}
\DeclareRobustCommand{\endProtected}[1]{\end{#1}}
\newcommand{\dbr}[1]{d_{\mbox{#1BR}}}
\newtheorem{lemma}{Lemma}
\newtheorem*{corollary}{Corollary}
\newtheorem{theorem}{Theorem}
\newtheorem{proposition}{Proposition}
\theoremstyle{definition}
\newtheorem{define}{Definition}
\newcommand{\column}[2]{
\left( \begin{array}{ccc}
#1 \\
#2
\end{array} \right)}

\newdimen\digitwidth
\settowidth\digitwidth{0}
\def~{\hspace{\digitwidth}}

\setlength{\parskip}{1pc}
\setlength{\parindent}{0pt}
\setlength{\topmargin}{-3pc}
\setlength{\textheight}{9.0in}
\setlength{\oddsidemargin}{0pc}
\setlength{\evensidemargin}{0pc}
\setlength{\textwidth}{6.5in}
\newcommand{\answer}[1]{\newpage\noindent\framebox{\vbox{{\bf CSCI 4273 Fall 2014} 
\hfill {\bf \name} \vspace{-1cm}
\begin{center}{Homework \#3}\end{center} } }\bigskip }

%absolute value code
\DeclarePairedDelimiter\abs{\lvert}{\rvert}%
\DeclarePairedDelimiter\norm{\lVert}{\rVert}
\makeatletter
\let\oldabs\abs
\def\abs{\@ifstar{\oldabs}{\oldabs*}}
%
\let\oldnorm\norm
\def\norm{\@ifstar{\oldnorm}{\oldnorm*}}
\makeatother

\def\dbar{{\mathchar'26\mkern-12mu d}}
\def \Frac{\displaystyle\frac}
\def \Sum{\displaystyle\sum}
\def \Int{\displaystyle\int}
\def \Prod{\displaystyle\prod}
\def \P[x]{\Frac{\partial}{\partial x}}
\def \D[x]{\Frac{d}{dx}}
\newcommand{\PD}[2]{\frac{\partial#1}{\partial#2}}
\newcommand{\PF}[1]{\frac{\partial}{\partial#1}}
\newcommand{\DD}[2]{\frac{d#1}{d#2}}
\newcommand{\DF}[1]{\frac{d}{d#1}}
\newcommand{\fix}[2]{\left(#1\right)_#2}
\newcommand{\ket}[1]{|#1\rangle}
\newcommand{\bra}[1]{\langle#1|}
\newcommand{\braket}[2]{\langle #1 | #2 \rangle}
\newcommand{\bopk}[3]{\langle #1 | #2 | #3 \rangle}
\newcommand{\Choose}[2]{\displaystyle {#1 \choose #2}}
\newcommand{\proj}[1]{\ket{#1}\bra{#1}}
\def\del{\vec{\nabla}}
\newcommand{\avg}[1]{\langle#1\rangle}
\newcommand{\piecewise}[4]{\left\{\beginProtected{array}{rl}#1&:#2\\#3&:#4\endProtected{array}\right.}
\newcommand{\systeme}[2]{\left\{\beginProtected{array}{rl}#1\\#2\endProtected{array}\right.}
\def \KE{K\!E}
\def\Godel{G$\ddot{\mbox{o}}$del}

\onehalfspacing

\begin{document}

\answer{}

\textbf{1:} In the 4B/5B encoding (see Table 2.4), only two of the 5-bit codes used end
in two 0s. How many possible 5-bit sequences are there (used by the existing
code or not) that meet the stronger restriction of having at most one leading
and at most one trailing 0? Could all 4-bit sequences be mapped to such 5-bit
sequences?

\textbf{2:} Show the 4B/5B encoding, and the resulting NRZI signal, for the following
bit sequence: 1110 0101 0000 0011

\textbf{3:} Assuming a framing protocol that uses bit stuffing, show the bit sequence
transmitted over the link when the frame contains the following bit sequence: 110101111101011111101011111110
Mark the stuffed bits.

\textbf{4:} Suppose that one byte in a buffer covered by the Internet checksum algorithm
needs to be decremented (e.g., a header hop count field). Give an algorithm
to compute the revised checksum without rescanning the entire buffer. Your
algorithm should consider whether the byte in question is low order or high
order

\textbf{5:} Suppose we want to transmit the message 11100011 and protect it from errors
using the CRC polynomial $x^3 + 1$.
\begin{enumerate}[(a)]
  \item{}Use polynomial long division to determine the message that should be
  transmitted\\
  \item{}Suppose the leftmost bit of the message is inverted due to noise on the
  transmission link. What is the result of the receiver’s CRC calculation?
  How does the receiver know that an error has occurred?\\
\end{enumerate}

\textbf{6:} Suppose you are designing a sliding window protocol for a 1-Mbps point-to-point
link to the moon, which has a one-way latency of 1.25 seconds. Assuming
that each frame carries 1 KB of data, what is the minimum number of bits you
need for the sequence number?

\textbf{7:} Suppose that we attempt to run the sliding window algorithm with SWS =
RWS = 3 and with MaxSeqNum = 5. The $N$th packet DATA[$N$] thus actually contains
N mod 5 in its sequence number field. Give an example in
which the algorithm becomes confused, that is, a scenario in which the receiver
expects DATA[5] and accepts DATA[0]—which has the same transmitted sequence number—in
its stead. No packets may arrive out of order. Note this
implies MaxSeqNum $\geq 6$ is necessary as well as sufficient.

\textbf{8:} Suppose the round-trip propagation delay for Ethernet is $46.4 \mu s$. This yields
a minimum packet size of 512 bits (464 bits corresponding to propagation
delay + 48 bits of jam signal).
\begin{enumerate}[(a)]
  \item{}What happens to the minimum packet size if the delay time is held constant,
    and the signaling rate rises to 100 Mbps?\\
  \item{}What are the drawbacks to so large a minimum packet size?\\
  \item{}If compatibility were not an issue, how might the specifications be written
so as to permit a smaller minimum packet size?\\
\end{enumerate}

\textbf{9:} How can a wireless node interfere with the communications of another node
when the two nodes are separated by a distance greater than the transmission
range of either node?

\textbf{10:} How can hidden terminals be detected in
802.11 networks?

\textbf{11:} Using the example network given in Figure 3.30, give the virtual circuit
tables for all the switches after each of the following connections is established.
Assume that the sequence of connections is cumulative, that is, the first connection
is still up when the second connection is established, and so on. Also
assume that the VCI assignment always picks the lowest unused VCI on each
link, starting with 0.
\begin{enumerate}[(a)]
  \item{}Host A connects to host B.\\
  \item{}Host C connects to host G.\\
  \item{}Host E connects to host I.\\
  \item{}Host D connects to host B.\\
  \item{}Host F connects to host J.\\
  \item{} Host H connects to host A.\\
\end{enumerate}

\textbf{12:} Give forwarding tables for switches S1–S4 in Figure 3.32. Each switch should
have a “default” routing entry, chosen to forward packets with unrecognized destination
addresses toward OUT. Any specific-destination table entries duplicated by the default
entry should then be eliminated.

\textbf{13:} Consider hosts X, Y, Z, W and learning bridges B1, B2, B3, with initially
empty forwarding tables, as in Figure 3.36.
\begin{enumerate}[(a)]
  \item{}Suppose X sends to Z. Which bridges learn where X is? Does Y’s network
  interface see this packet?\\
  \item{}Suppose Z now sends to X. Which bridges learn where Z is? Does Y’s
  network interface see this packet?\\
  \item{}Suppose Z now sends to X. Which bridges learn where Z is? Does Y’s
  network interface see this packet?\\
  \item{}Finally, suppose Z sends to Y. Which bridges learn where Z is? Does W’s
  network interface see this packet?\\
\end{enumerate}

\textbf{14:} Suppose two learning bridges B1 and B2 form a loop as shown in Figure 3.38,
and do \emph{not} implement the spanning tree algorithm. Each bridge maintains a
single table of <\emph{address}, \emph{interface}> pairs.
\begin{enumerate}[(a)]
  \item{}What will happen if M sends to L?\\
  \item{}Suppose a short while later L replies to M. Give a sequence of events that
leads to one packet from M and one packet from L circling the loop in
opposite directions.\\
\end{enumerate}


\end{document}
