\documentclass[12pt, notitlepage, final]{article} 

\newcommand{\name}{Vince Coghlan}

\usepackage{amsfonts}
\usepackage{amssymb}
\usepackage{amsmath}
\usepackage{latexsym}
\usepackage{enumerate}
\usepackage{amsthm}
\usepackage{nccmath}
\usepackage{setspace}
\usepackage[pdftex]{graphicx}
\usepackage{epstopdf}
\usepackage[siunitx]{circuitikz}
\usepackage{tikz}
\usepackage{float}
\usepackage{cancel}
\usepackage{pgfplots}
\usepackage{setspace}
\usepackage{overpic}
\usepackage{mathtools}
\usepackage{listings}
\usepackage{color}

\numberwithin{equation}{section}
\DeclareRobustCommand{\beginProtected}[1]{\begin{#1}}
\DeclareRobustCommand{\endProtected}[1]{\end{#1}}
\newcommand{\dbr}[1]{d_{\mbox{#1BR}}}
\newtheorem{lemma}{Lemma}
\newtheorem*{corollary}{Corollary}
\newtheorem{theorem}{Theorem}
\newtheorem{proposition}{Proposition}
\theoremstyle{definition}
\newtheorem{define}{Definition}
\newcommand{\column}[2]{
\left( \begin{array}{ccc}
#1 \\
#2
\end{array} \right)}

\newdimen\digitwidth
\settowidth\digitwidth{0}
\def~{\hspace{\digitwidth}}

\setlength{\parskip}{1pc}
\setlength{\parindent}{0pt}
\setlength{\topmargin}{-3pc}
\setlength{\textheight}{9.0in}
\setlength{\oddsidemargin}{0pc}
\setlength{\evensidemargin}{0pc}
\setlength{\textwidth}{6.5in}
\newcommand{\answer}[1]{\newpage\noindent\framebox{\vbox{{\bf CSCI 4273 Fall 2014} 
\hfill {\bf \name} \vspace{-1cm}
\begin{center}{Homework \#1}\end{center} } }\bigskip }

%absolute value code
\DeclarePairedDelimiter\abs{\lvert}{\rvert}%
\DeclarePairedDelimiter\norm{\lVert}{\rVert}
\makeatletter
\let\oldabs\abs
\def\abs{\@ifstar{\oldabs}{\oldabs*}}
%
\let\oldnorm\norm
\def\norm{\@ifstar{\oldnorm}{\oldnorm*}}
\makeatother

\def\dbar{{\mathchar'26\mkern-12mu d}}
\def \Frac{\displaystyle\frac}
\def \Sum{\displaystyle\sum}
\def \Int{\displaystyle\int}
\def \Prod{\displaystyle\prod}
\def \P[x]{\Frac{\partial}{\partial x}}
\def \D[x]{\Frac{d}{dx}}
\newcommand{\PD}[2]{\frac{\partial#1}{\partial#2}}
\newcommand{\PF}[1]{\frac{\partial}{\partial#1}}
\newcommand{\DD}[2]{\frac{d#1}{d#2}}
\newcommand{\DF}[1]{\frac{d}{d#1}}
\newcommand{\fix}[2]{\left(#1\right)_#2}
\newcommand{\ket}[1]{|#1\rangle}
\newcommand{\bra}[1]{\langle#1|}
\newcommand{\braket}[2]{\langle #1 | #2 \rangle}
\newcommand{\bopk}[3]{\langle #1 | #2 | #3 \rangle}
\newcommand{\Choose}[2]{\displaystyle {#1 \choose #2}}
\newcommand{\proj}[1]{\ket{#1}\bra{#1}}
\def\del{\vec{\nabla}}
\newcommand{\avg}[1]{\langle#1\rangle}
\newcommand{\piecewise}[4]{\left\{\beginProtected{array}{rl}#1&:#2\\#3&:#4\endProtected{array}\right.}
\newcommand{\systeme}[2]{\left\{\beginProtected{array}{rl}#1\\#2\endProtected{array}\right.}
\def \KE{K\!E}
\def\Godel{G$\ddot{\mbox{o}}$del}

\onehalfspacing

\begin{document}

\answer{}

\textbf{1:} first I tried "whois princeton.edu" and got:

\begin{lstlisting}
Domain Name: PRINCETON.EDU

Registrant:
   Princeton University
   Office of Information Technology
   701 Carnegie Center, Suite 302
   Princeton, NJ 08540
   UNITED STATES

Administrative Contact:
   Princeton  University
   Contact for Internet name and number resources
   Princeton University
   Office of Information Technology
   701 Carnegie Center, Suite 302
   Princeton, NJ 08540
   UNITED STATES
   (609) 258-8700
   netmaster@princeton.edu

Technical Contact:
   Princeton  University
   Contact for Internet name and number resources
   Princeton University
   Office of Information Technology
   701 Carnegie Center, Suite 302
   Princeton, NJ 08540
   UNITED STATES
   (609) 258-8700
   netmaster@princeton.edu

Name Servers:
   DNS.PRINCETON.EDU           128.112.129.15
   DIKAHBLE.PRINCETON.EDU      128.112.134.4
   NS1.FAST.NET
   NS2.FAST.NET
   ADNS1.UCSC.EDU
   ADNS2.UCSC.EDU

Domain record activated:    03-Apr-1987
Domain record last updated: 05-Oct-2010
Domain expires:             31-Jul-2015
\end{lstlisting}
Then for "whois princeton" I got a list of domains that the search feature found.

\textbf{2:}
\begin{enumerate}[(a)]
  \item{}The total time to receive all bytes is the handshake, the transmit time, and
    time it takes to propegate.  This is:
    \[
      2\cdot100\text{ms} + 1000\text{KB}/1.5\text{Mbps} + \frac{1}{2}100\text{ms} = 5.71 \text{ seconds}
    \]
  \item{}We will be sending 1000 packets ($1000\text{KB}/1\text{KB}$).  This means there will be an entire
    RTT for each packet, or:
    \[
      5.71\text{seconds} + 1000\cdot100\text{ms} = 105.71\text{ seconds}
    \]
  \item{}We will be sending data $1000/20 = 50$ times. This means:
    \[
      \text{time} = 2\cdot100\text{ms} + 50\cdot{100}\text{ms} + \frac{1}{2}100\text{ms} = 5.25 \text{ seconds}
    \]
  \item{}To find out how many times we must send, we can look at the sum:
    \[
      \sum_{k=1}^n{2^{k-1}}
    \]
    when n is 10, this number is 1023, the closest thing above 1000.  This means that we
    will need to send 10 times. Our time ends up being:
    \[
      2\cdot100\text{ms} + 10\cdot{100}\text{ms} + \frac{1}{2}100\text{ms} = 1.25 \text{ seconds}
    \]
\end{enumerate}

\textbf{3:} Multicast is more useful than unicast if you have many recipients.  You only
need to send the data once (or as much as the bandwidth will let you) instead of having to
send data once for every target machine.  It is better than broadcast in some circumstances
if you dont want your data to go to anyone.

\textbf{4:} STDM is useful for the phone system since when you are listening to a voice,
you want to be able to hear all the frequency ranges.  You wouldnt want to be limited to what
you could hear and have to listen to different people in different frequencies.  FDM is
good for broadcast networks since the channels are always on, and you know that each one will
want to communicate at the same time.  If you want to add a channel to television, you just
allocate another slot.  Neither of these methods are good for a general purpose
network, since you would need to allocate a time or frequency unit to every single line, and
you are not sure if they are going to send or not at any given time.

\textbf{5:}
\begin{enumerate}[(a)]
  \item{}
    \[
      \text{RTT} = \frac{2\cdot\text{distance}}{\text{speed}} = \frac{2\cdot 385000000}{3\cdot10^8} = 2.6 \text{ seconds}
    \]
  \item{}
    \[
      \text{delay}\times\text{bandwidth} = 2.6\cdot1Gbps = 2.6Gb = 320.8 \text{ GB}
    \]
  \item{} This is the maximum amount of data that can be on the network at any given time.
  \item{}
    \[
      \text{time} = \text{RTT} + \frac{\text{size}}{\text{bandwidth}} = 2.6 + (\frac{25\cdot8}{100}) = 4.6 \text{ seconds}
    \]
\end{enumerate}

\textbf{6:}
\begin{enumerate}[(a)]
  \item{} The effective bandwidth is 100 Mbps, since thats how fast the switches are.  The
    switches are just sending the data down the line.
  \item{} The delay due to bandwidth is 12000 bits/100MBps, or $120\mu$s.  The total delay
    includes propegation delay of 10.  With three switches, like in this setup, the delay will
    be:
    \[
      4\cdot120\mu\text{s} + 4\cdot10\mu\text{s} = 520\mu\text{s}
    \]
    The acknowledgement will take half a microsecond per link to be sent.  this means that
    the delay is:
    \[
      4\cdot0.5\mu\text{s} + 4\cdot10\mu\text{s} = 42\mu\text{s}
    \]
    for a grand total of $562\mu$s.  12000 bits in $562\mu$s is 21.35Mbps.
  \item{}
    \[
      \frac{100\cdot4.7}{12\text{ hours}} = 93.46\text{ Mbps}
    \]
\end{enumerate}

\textbf{7:}
\begin{enumerate}[(a)]
  \item{}Setting the transmission time equations together we find:
\[
  \frac{0.5\text{MB}}{t-1} = \frac{0.4\text{MB}}{t-2}\Rightarrow t=6
\]
meaning that the bandwidth is:
\[
  \frac{0.5\text{MB}}{5} = 0.84\text{ Mbps}
\]
\item{} Since both messages are sent through the same channel, they have the
  same latency.
\end{enumerate}

\textbf{9:} cs.princeton.edu would not respond to my pings, but princeton.edu had a
ping of around 60ms and cisco.com had a ping of about 12ms.  The reason for the descrepency
can be seen if you type in traceroute.  Princeton is sending the traffic through more than
twice as many places as cisco is.  Also cisco is a private corperation who can afford better
equipment than an educational institution. The output is:
\begin{lstlisting}
PING www.princeton.edu (128.112.132.86): 56 data bytes
64 bytes from 128.112.132.86: icmp_seq=0 ttl=47 time=60.996 ms
64 bytes from 128.112.132.86: icmp_seq=1 ttl=47 time=60.056 ms
64 bytes from 128.112.132.86: icmp_seq=2 ttl=47 time=67.742 ms
64 bytes from 128.112.132.86: icmp_seq=3 ttl=47 time=59.036 ms
^C
--- www.princeton.edu ping statistics ---
4 packets transmitted, 4 packets received, 0.0% packet loss
round-trip min/avg/max/stddev = 59.036/61.957/67.742/3.411 ms


PING e144.dscb.akamaiedge.net (23.4.128.211): 56 data bytes
64 bytes from 23.4.128.211: icmp_seq=0 ttl=58 time=11.128 ms
64 bytes from 23.4.128.211: icmp_seq=1 ttl=58 time=11.017 ms
64 bytes from 23.4.128.211: icmp_seq=2 ttl=58 time=13.449 ms
^C
--- e144.dscb.akamaiedge.net ping statistics ---
3 packets transmitted, 3 packets received, 0.0% packet loss
round-trip min/avg/max/stddev = 11.017/11.865/13.449/1.121 ms
\end{lstlisting}
I cant measure this at different times of the day since I am doing the assignment
minutes before it is due.

\textbf{10:} We have already seen in the last problem how more servers means a higher
RTT.  We can see the output from some sites below.

\begin{lstlisting}
traceroute: Warning: www.google.com has multiple addresses; using 74.125.225.211
traceroute to www.google.com (74.125.225.211), 64 hops max, 52 byte packets
 1  192.168.0.1 (192.168.0.1)  2.016 ms  1.307 ms  0.981 ms
 2  c-71-196-136-1.hsd1.co.comcast.net (71.196.136.1)  16.265 ms  8.405 ms  10.199 ms
 3  te-8-2-ur01.boulder.co.denver.comcast.net (68.86.129.229)  11.615 ms  8.983 ms  10.040 ms
 4  te-7-4-ur02.boulder.co.denver.comcast.net (68.86.103.122)  10.329 ms  9.982 ms  10.027 ms
 5  xe-13-3-1-0-ar01.aurora.co.denver.comcast.net (68.86.179.81)  12.549 ms  8.700 ms  11.375 ms
 6  te-0-1-0-4-cr01.chicago.il.ibone.comcast.net (68.86.95.201)  14.244 ms
    he-3-10-0-0-cr01.denver.co.ibone.comcast.net (68.86.92.25)  12.609 ms
    68.86.87.89 (68.86.87.89)  15.816 ms
 7  xe-5-1-0-0-pe01.910fifteenth.co.ibone.comcast.net (68.86.82.206)  12.122 ms  10.825 ms  11.702 ms
 8  as15169-1-c.910fifteenth.co.ibone.comcast.net (23.30.206.106)  25.539 ms  25.615 ms  20.508 ms
 9  72.14.234.57 (72.14.234.57)  12.474 ms  12.747 ms  10.903 ms
10  209.85.251.111 (209.85.251.111)  12.827 ms  13.359 ms  17.532 ms
11  den03s06-in-f19.1e100.net (74.125.225.211)  9.317 ms  12.696 ms  12.408 ms

traceroute to www.colorado.edu (128.138.129.98), 64 hops max, 52 byte packets
 1  192.168.0.1 (192.168.0.1)  1.103 ms  1.300 ms  1.034 ms
 2  c-71-196-136-1.hsd1.co.comcast.net (71.196.136.1)  10.144 ms  9.291 ms  8.990 ms
 3  te-8-2-ur01.boulder.co.denver.comcast.net (68.86.129.229)  10.006 ms  8.746 ms  9.860 ms
 4  xe-13-3-1-0-ar01.denver.co.denver.comcast.net (68.86.103.157)  10.689 ms  10.992 ms  9.444 ms
 5  te-4-8-ur01.denver.co.denver.comcast.net (68.86.104.106)  12.347 ms  11.723 ms  10.915 ms
 6  68.86.128.18 (68.86.128.18)  10.155 ms  11.554 ms  10.180 ms
 7  ucb-i1-frgp.colorado.edu (198.59.55.9)  10.265 ms  10.653 ms  12.161 ms
 8  fw-juniper.colorado.edu (128.138.81.194)  12.178 ms *  12.508 ms
 9  hut-fw.colorado.edu (128.138.81.249)  14.506 ms  12.155 ms  11.904 ms
10  comp-hut.colorado.edu (128.138.81.11)  12.249 ms  11.552 ms  11.517 ms
11  www.colorado.edu (128.138.129.98)  11.603 ms  12.122 ms  11.366 ms

traceroute to thepiratebay.se (194.71.107.27), 64 hops max, 52 byte packets
 1  192.168.0.1 (192.168.0.1)  1.800 ms  1.687 ms  1.463 ms
 2  c-71-196-136-1.hsd1.co.comcast.net (71.196.136.1)  8.998 ms  9.312 ms  10.313 ms
 3  te-8-2-ur01.boulder.co.denver.comcast.net (68.86.129.229)  8.880 ms  9.171 ms  9.746 ms
 4  te-7-4-ur02.boulder.co.denver.comcast.net (68.86.103.122)  9.636 ms  9.087 ms  9.349 ms
 5  xe-13-3-2-0-ar01.aurora.co.denver.comcast.net (68.86.179.97)  11.518 ms  11.680 ms  10.332 ms
 6  he-3-4-0-0-cr01.denver.co.ibone.comcast.net (68.86.90.149)  15.268 ms
    he-3-9-0-0-cr01.denver.co.ibone.comcast.net (68.86.92.21)  13.843 ms
    68.86.87.89 (68.86.87.89)  15.150 ms
 7  be-16-pe04.ashburn.va.ibone.comcast.net (68.86.84.226)  28.342 ms  26.085 ms  28.247 ms
 8  he-0-12-0-0-pe03.1950stemmons.tx.ibone.comcast.net (68.86.86.178)  28.485 ms  25.271 ms  26.069 ms
 9  ae-19.r07.dllstx09.us.bb.gin.ntt.net (129.250.66.29)  26.523 ms  25.708 ms  25.435 ms
10  * * *
11  ae-3.r20.asbnva02.us.bb.gin.ntt.net (129.250.3.50)  59.842 ms  59.072 ms  58.726 ms
12  * ae-0.r21.asbnva02.us.bb.gin.ntt.net (129.250.4.5)  60.351 ms *
13  * * ae-2.r23.amstnl02.nl.bb.gin.ntt.net (129.250.2.145)  195.423 ms
14  ae-2.r02.amstnl02.nl.bb.gin.ntt.net (129.250.2.159)  270.771 ms  307.288 ms  194.190 ms
15  xe-4-1.r02.dsdfge01.de.bb.gin.ntt.net (129.250.2.65)  155.367 ms  147.905 ms  152.986 ms
16  * * *
17  * * *
18  * * *
19  * * *
20  * * *
21  * * *
22  * * *
23  * * *
24  * * *
(And it just keeps going like that, lol piratebay)

traceroute to localhost (127.0.0.1), 64 hops max, 52 byte packets
 1  localhost (127.0.0.1)  0.133 ms  0.029 ms  0.027 ms
\end{lstlisting}


Pinging myself took no time at all!

\end{document}
